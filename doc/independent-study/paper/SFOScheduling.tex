\documentclass[letterpaper, 10 pt, conference]{ieeeconf}
\usepackage{fullpage}
\usepackage{amsmath,amsfonts,amssymb}
\usepackage{verbatim}
\usepackage{graphicx}
\usepackage{booktabs}
\usepackage{multirow}

\usepackage{xargs}
\usepackage{xspace}
\IEEEoverridecommandlockouts                             
\overrideIEEEmargins   
% Insert comments
%\usepackage[colorinlistoftodos,prependcaption,textsize=tiny]{todonotes}
%\newcommandx{\kasper}[2][1=]{\todo[linecolor=blue,backgroundcolor=blue!25,bordercolor=blue,#1]{Kasper: #2}}
\newcommand{\tool}[0]{\textsc{TOOLNAME}\xspace} 
%\title{Uncertainty Management for Airport Surface Planning and Scheduling}
\title{Analysis of Airport Surface Operations under Uncertainty}
\author{Robert Morris$^{1}$ \and Heron Yang$^{2}  \and Corina Pasareanu$^{3}$ \\ % <-this % stops a space
\thanks{$^{3}$ Heron Yang {\tt\small heronyang@cmu.edu}} \\
\thanks{$^{1}$ Robert Morris {\tt\small robert.a.morris@nasa.gov}} \\%
\thanks{$^{2}$ Corina Pasareanu {\tt\small corina.pasareanu@west.cmu.edu}}%
}
\begin{document}
\maketitle
\begin{abstract}
\end{abstract}
\section{Introduction and Motivation}

Se study the problem of planning and scheduling in the context of real-world uncertainty and create a generic airport simulation tool that is easily extensible to different scenarios and airports. We explore different scheduling methods  and also created an uncertainty-aware scheduler which produces roust schedules with simulated uncertainty. Lastly, we built an uncertainty module to model real-word uncertainty and evaluate the robustness of scheduler in light of different scenarios. The work is open sourced at \url{https://github.com/heronyang/airport-simulation}.

In this paper we focus on analyzing the relationship between tightness and performance factors of a scheduler: we expect to see that a scheduler with low tightness brings better performance, vice versa. 
Furthermore, we study how uncertainty affects the decisions made by a scheduler within our simulation environment, based on the data from the San Francisco Airport. We’ve observed how different amount of uncertainty generates conflicts on an airport surface, and it’s planned to implement a scheduler that refers to predictions with uncertainties generated by a simulator while scheduling. We expect to see a relationship between the uncertainty sensitivity and the real performance.



\section{Problem}


\section{Summary of Previous Work}
\section{Approach}
\section{System}
\section{Experiments}
\section{Future Work}
\section{Summary}
\bibliographystyle{IEEEtran}
\bibliography{IEEEabrv,icra,references}
\end{document}
