\documentclass[letterpaper, 10 pt, conference]{ieeeconf}
\usepackage{fullpage}
\usepackage{amsmath,amsfonts,amssymb}
\usepackage{verbatim}
\usepackage{graphicx}
\usepackage{booktabs}
\usepackage{multirow}
\usepackage[hyphens]{url}

\usepackage{xargs}
\usepackage{xspace}
\IEEEoverridecommandlockouts                             
\overrideIEEEmargins   
% Insert comments
%\usepackage[colorinlistoftodos,prependcaption,textsize=tiny]{todonotes}
%\newcommandx{\kasper}[2][1=]{\todo[linecolor=blue,backgroundcolor=blue!25,bordercolor=blue,#1]{Kasper: #2}}
\newcommand{\tool}[0]{\textsc{TOOLNAME}\xspace} 
%\title{Uncertainty Management for Airport Surface Planning and Scheduling}
\title{Analysis of Airport Surface Operations under Uncertainty}
\author{Robert Morris$^{1}$ \and Heron Yang$^{2}$  \and Corina Pasareanu$^{3}$ \\ % <-this % stops a space
\thanks{$^{3}$ Heron Yang {\tt\small heronyang@cmu.edu}} \\
\thanks{$^{1}$ Robert Morris {\tt\small robert.a.morris@nasa.gov}} \\%
\thanks{$^{2}$ Corina Pasareanu {\tt\small corina.pasareanu@west.cmu.edu}}%
}
\begin{document}
\maketitle
\begin{abstract}
\end{abstract}
\section{Introduction and Motivation}

Se study the problem of planning and scheduling in the context of real-world uncertainty and create a generic airport simulation tool that is easily extensible to different scenarios and airports. We explore different scheduling methods and also created an uncertainty-aware scheduler which produces roust schedules with simulated uncertainty. Lastly, we built an uncertainty module to model real-word uncertainty and evaluate the robustness of scheduler in light of different scenarios. The work is open sourced at \url{https://github.com/heronyang/airport-simulation}.

In this paper we focus on analyzing the relationship between tightness and performance factors of a scheduler: we expect to see that a scheduler with low tightness brings better performance, vice versa. Furthermore, we study how uncertainty affects the decisions made by a scheduler within our simulation environment, based on the data from the San Francisco Airport. We've observed how different amount of uncertainty generates conflicts on an airport surface, and it's planned to implement a scheduler that refers to predictions with uncertainties generated by a simulator while scheduling. We expect to see a relationship between the uncertainty sensitivity and the real performance.

\section{Problem}

1. Describe the how common uncertainties happen in airport surfaces and how important it is to handle them

2. Describe the importance of reducing the taxi-time (economically, environmentally) and delays (economically)

3. Describe the current schedulers only deal with uncertainties by consistently rescheduling. The downside of this is: a) we have no confident or knowledge in how well a generated schedule can perform under uncertainty b) since the reschedule window is short, the computation can be executed is limited

\section{Summary of Previous Work}

\textbf{MILP}

Hanbong Lee's Thesis

\textbf{chance constrained scheduling}

chance-constrained scheduling via conflict-directed risk allocation (Andrew J. Wang)

air traffic flow management under uncertainty: application of chance constraints (Dr Gillian Clare)

\section{Approach}

Describe the design of the uncertainty-aware scheduler

\section{System}

\url{https://docs.google.com/presentation/d/1QpV9yJCrbKvzwpE17w0m9_30UcFdJAJVOLPI4aZuNl8/edit#slide=id.g34bb45218b_0_8}

\subsection{Domain}

 (page 3)

input: surface data, parameters

simulation: constraint, uncertainty, scheduler

scheduler: options, output

metrics: makespan, taxi-time, conflicts, delays, queue size

\subsection{Flow}

(use and describe page 4)

cost function calculation (not sure)

schedulers: deterministic scheduler, uncertainty-aware scheduler

\section{Experiments}

\subsection{Dataset}

1. Simple data

2. SFO Terminal 2: describe the data source

\subsection{Environment Setup}

- (control variables) how many runs we've sampled per experiment

- (control variables) describe all other fixed/predefined variables

- (experimental variables) three different uncertainties (three pairs of \% of unexpected delay at node and the delay time)

- (experimental variables) each experiment below, we apply both a) the deterministic scheduler and b) the uncertainty-aware scheduler

- (output metrics) describe what metrics we are observing/collecting

\subsection{Uncertainty Experiments}

1. uncertainty (small, medium, large) v.s. number of conflicts

2. uncertainty (small, medium, large) v.s. makespan

3. uncertainty (small, medium, large) v.s. delay

4. uncertainty (small, medium, large) v.s. queue size

\subsection{Reschedule Time Experiments}

1. reschedule time (resolve conflict time) v.s. number of conflicts

2. reschedule time (resolve conflict time) v.s. makespan

3. reschedule time (resolve conflict time) v.s. delay

4. reschedule time (resolve conflict time) v.s. queue size

\subsection{Execution Cost}

1. how long it takes for above experiments, and compare between the schedulers

2. how long it takes for running on different size of dataset

\section{Future Work}
\section{Summary}
\bibliographystyle{IEEEtran}
\bibliography{IEEEabrv,icra,references}
\end{document}
