\documentclass[]{article} \usepackage[margin=1.3in]{geometry} \setlength{\columnsep}{0.2in} \pagenumbering{gobble}
\usepackage{lmodern}
\usepackage{amssymb,amsmath}
\usepackage{ifxetex,ifluatex}
\usepackage{fixltx2e} % provides \textsubscript
\ifnum 0\ifxetex 1\fi\ifluatex 1\fi=0 % if pdftex
  \usepackage[T1]{fontenc}
  \usepackage[utf8]{inputenc}
\else % if luatex or xelatex
  \ifxetex
    \usepackage{mathspec}
  \else
    \usepackage{fontspec}
  \fi
  \defaultfontfeatures{Ligatures=TeX,Scale=MatchLowercase}
\fi
% use upquote if available, for straight quotes in verbatim environments
\IfFileExists{upquote.sty}{\usepackage{upquote}}{}
% use microtype if available
\IfFileExists{microtype.sty}{%
\usepackage[]{microtype}
\UseMicrotypeSet[protrusion]{basicmath} % disable protrusion for tt fonts
}{}
\PassOptionsToPackage{hyphens}{url} % url is loaded by hyperref
\usepackage[unicode=true]{hyperref}
\hypersetup{
            pdfborder={0 0 0},
            breaklinks=true}
\urlstyle{same}  % don't use monospace font for urls
\IfFileExists{parskip.sty}{%
\usepackage{parskip}
}{% else
\setlength{\parindent}{0pt}
\setlength{\parskip}{6pt plus 2pt minus 1pt}
}
\setlength{\emergencystretch}{3em}  % prevent overfull lines
\providecommand{\tightlist}{%
  \setlength{\itemsep}{0pt}\setlength{\parskip}{0pt}}
\setcounter{secnumdepth}{0}
% Redefines (sub)paragraphs to behave more like sections
\ifx\paragraph\undefined\else
\let\oldparagraph\paragraph
\renewcommand{\paragraph}[1]{\oldparagraph{#1}\mbox{}}
\fi
\ifx\subparagraph\undefined\else
\let\oldsubparagraph\subparagraph
\renewcommand{\subparagraph}[1]{\oldsubparagraph{#1}\mbox{}}
\fi

% set default figure placement to htbp
\makeatletter
\def\fps@figure{htbp}
\makeatother


\date{}

\begin{document}

\section{Schedule under Uncertainty}\label{schedule-under-uncertainty}

\begin{itemize}
\tightlist
\item
  \(P_r\): the possibility of moving to the next node at a node in the
  real world
\item
  \(P_p\): the possibility of moving to the next node at a node in the
  prediction
\item
  \(Cost_p(schedule)\): cost function of a schedule, used for evaluating
  a schedule while scheduling
\item
  \(Cost_r(simulation)\): cost function of a simulation run, used for
  evaluating a scheduler at the end of a simulation
\end{itemize}

\subsection{Approach \#1 (Robert)}\label{approach-1-robert}

Resolves the conflicts (adds delay to the aircraft with lower priority)
that happens with possibility \(P_c\) larger than the predefined
threshold \(P_{threshold}\). \(P_{threshold}\) can be set to adjust what
kind of scheduler we prefer, a more efficient one or a more robust one.\\
\textbf{Challenge}: what is the programming model of calculating the
probability combination of conflicts?

\subsection{Approach \#2 (Corina)}\label{approach-2-corina}

We start with a deterministic greedy schedule. Then, we run predictions
for N (say 1000) rounds. In each prediction, we apply a deterministic
conflict resolution method to deal with conflicts. Therefore, we will
obtain N schedules. In these N schedules, we pick the one with a minimum
cost. The cost should be defined as \$Cost(schedule, frequency of this
schedule).\\\textbf{Challenge}: we are not sure how to model frequency
into the cost function.

\subsection{Approach \#3 (Heron)}\label{approach-3-heron}

We start with a deterministic greedy schedule. Then, we run predictions
for N (say 1000) rounds. In each prediction, we counts the conflicts but
we don't resolve them. In the first iteration, we plot the cost (which
counts the conflict as a penalty) and got a list of conflicts with
corresponding frequency. We resolve the conflict with a highest
frequency, then run the same iteration again util we find a minimum
cost.\\\textbf{Challenge}: this is the most computation heavy solution
and we are not sure if this will provide a good schedule.

\end{document}
